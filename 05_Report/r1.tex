\documentclass[conference]{IEEEtran}
%\IEEEoverridecommandlockouts
% The preceding line is only needed to identify funding in the first footnote. If that is unneeded, please comment it out.
\usepackage{cite}
\usepackage{amsmath,amssymb,amsfonts}
\usepackage{algorithmic}
\usepackage{graphicx}
\usepackage{textcomp}
\usepackage{xcolor}
\usepackage{siunitx}
\usepackage{float}
\usepackage{tabularx} % For adjustable table widths
\usepackage{booktabs} % For better table formatting
\usepackage{hyperref}
\hypersetup{breaklinks=true}
\usepackage{url}


\def\BibTeX{{\rm B\kern-.05em{\sc i\kern-.025em b}\kern-.08em
    T\kern-.1667em\lower.7ex\hbox{E}\kern-.125emX}}
\begin{document}

\title{Unbeatable Website on a Pi: A Self-Adaptive Approach\\
}

\author{\IEEEauthorblockN{Daniel Almeida}
    \IEEEauthorblockA{\textit{University of Waterloo} \\
        \textit{Electrical \& Computer Engineering}\\
        Student ID: 20986346 \\
        d2almeid@uwaterloo.ca}

    \and

    \IEEEauthorblockN{Arthur Li}
    \IEEEauthorblockA{\textit{University of Waterloo} \\
        \textit{Electrical \& Computer Engineering}\\
        Student ID: - \\
        -@uwaterloo.ca}
}

\maketitle

%\begin{abstract}
%This document is a model and instructions for \LaTeX.
%This and the IEEEtran.cls file define the components of your paper [title, text, heads, etc.]. *CRITICAL: Do Not Use Symbols, Special Characters, Footnotes, 
%or Math in Paper Title or Abstract.
%\end{abstract}

%\begin{IEEEkeywords}
%component, formatting, style, styling, insert
%\end{IEEEkeywords}

\section{\textbf{Introduction}}\label{intro}
First introduced in 2012, the Raspberry Pi (RPi) has since sold over 60 million units and sales continue to rise \cite{pi_sales}. While initially created to provide a more affordable option for students interested in computer science and the basics of coding, it became a go-to for teachers, creators and many DIY (do-it-yourself) projects such as "Magic Mirror", which integrates the RPi into a mirror and displays weather forecasts and emails \cite{magic_mirror}. As time went on, it was also the case that industrial applications of the RPi started gaining some popularity \cite{pi_history}.

With the RPi enabling makers with the ability to quickly prototype and launch systems and IoT (Internet of Things) projects at a low cost of entry, it quickly became apparent that its utility could potentially expand to more industrial applications. Benefits included cost-effectiveness, flexibility and customizability to specific use cases. In terms of cost for example, a direct comparison of a Raspberry Pi 4 8GB to AWS (Amazon Web Services) T2 Micro reveals that it would only cost 103.17 GBP using the RPi setup versus 191.52 GBP for the AWS setup \cite{pi_vs_AWS}. The RPi also allows for rapid prototyping and deployment, enabling small businesses and hobbyists to more easily experiment and take advantage of its open-source ecosystem \cite{pi_history}.

However, there are many issues often associated with self-hosting services on the RPi that limit its practicability in industrial settings. These include factors such as the need for regular backups to avoid data loss, a robust security implementation, frequent downtime due to software updates, lack of redundancy for critical applications where uptime and reliability are paramount (e.g. banking), power-related issues, overheating, limited or challenging scalability, and the challenge of network stability due to uncontrollable factors. The RPi's inherent design also means limited processing power and memory, which may be insufficient to support high traffic or computationally intensive applications \cite{pi_headaches,pi_website_hosting}. These many issues are addressed by commercial cloud services such as AWS, which do offer powerful computing resources, redundancy and scalability. And IBM mainframes, known for their reliability and survivability even under the most extreme events, could be seen as unbeatable solutions for banking servers and other mission-critical applications. For example, in April of 2024, 200 mainframes in IBM's Poughkeepsie, NY, facility were left unscathed after a 4.8 magnitude earthquake \cite{IBM_mainframes}, showcasing its advertised resiliency.

While commercial cloud services and mainframes do provide a robust solution to the many issues presented, the significant cost and complexity may make them inaccessible for DIY applications with limited budgets looking to provide high quality services, thus a novel approach is required. This leads us to the exploration of a self-adaptive solution; aiming to enhance RPi capabilities and achieve higher levels of system stability and reliability.

\section{\textbf{Proposed Solution}}\label{proposed_solution}
The presented solution explores how a self-adaptive RPi website hosting setup (RPiWeb) could bridge the gap between the aforementioned benefits to the RPi, the stability, reliability, security, and performance of traditional cloud services like AWS as well as the resiliency of IBM mainframes. That is, how a RPi could also—at a much smaller scale—be considered unbeatable. In many cases, users may opt to mitigate the issues by simply over-provisioning resources, but this approach, on top of the added manual labour, can quickly become costly and inefficient. Instead, a self-adaptive approach aims to equip the RPi with Self-CHOP (Self-Configuring, Self-Healing, Self-Optimizing, and Self-Protecting) capabilities:

\begin{itemize}
    \item \textbf{Self-Configuring:} The RPi is able to automatically adjust configuration based on workload demands. During peak traffic, the system may adjust CPU clock speeds, ...
    \item \textbf{Self-Healing:} When the RPi experiences a crash due to overload, a watchdog mechanism is able to reboot the system, minimizing downtime and removing the need for human intervention.
    \item \textbf{Self-Optimizing:} Dynamic adjustments to improve efficiency are made through content degradation, priority-based fallback pages, ...
    \item \textbf{Self-Protecting:} The SAS (Self-Adaptive System) should implement security measures to detect and block suspicious and/or unauthorized activity, maintaining a secure environment for the application and its data.
\end{itemize}

Ultimately, this will push the boundaries of what the RPi can achieve, making the RPi a more viable option for applications requiring reliability and resilience. The RPi would autonomously adapt to environmental and workload changes, making it more applicable to industrial IoT use cases and other critical, cost-sensitive applications, without sacrificing cost-effectiveness and minimal complexity in comparison.

To achieve this, 

\begin{thebibliography}{00}

    \bibitem{pi_sales} L. Pounder, "Raspberry Pi celebrates 12 years as sales break 61 million units," Tom's Hardware, Feb. 29, 2024. [Online]. Available: \url{https://www.tomshardware.com/raspberry-pi/raspberry-pi-celebrates-12-years-as-sales-break-61-million-units}. [Accessed: Dec. 4, 2024].

    \bibitem{magic_mirror} L. Upton, "Magic Mirror," Raspberry Pi Foundation, Apr. 29, 2014. [Online]. Available: \url{https://www.raspberrypi.com/news/magic-mirror/}. [Accessed: Dec. 4, 2024].

    \bibitem{pi_history} "Industrial Raspberry Pi: A brief history and current state," OnLogic, Mar. 14, 2024. [Online]. Available: \url{https://www.onlogic.com/blog/industrial-raspberry-pi-a-brief-history-and-current-state/}. [Accessed: Dec. 4, 2024].

    \bibitem{pi_headaches} https://www.makeuseof.com/raspberry-pi-difficulties-self-hosting-services/

    \bibitem{pi_website_hosting} https://fastercapital.com/content/Web-Server--Hosting-a-Website-with-Raspberry-Pi.html\#Troubleshooting-common-issues-when-hosting-a-website-with-Raspberry-Pi

    \bibitem{pi_vs_AWS} https://nelop.com/comparing-a-raspberry-pi-4-to-aws/

    \bibitem{IBM_mainframes} https://www.datacenterdynamics.com/en/news/ibm-mainframes-survive-48-magnitude-earthquake-in-us-east-coast/


\end{thebibliography}

\end{document}
